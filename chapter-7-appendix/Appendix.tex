\chapter{Appendices} %% change
\label{Appendices} %% change
\graphicspath{ {./Appendix/figures/} }  %% change
\captionsetup[figure]{labelfont=bf}
\captionsetup{margin=1.5em}
\captionsetup[table]{labelfont=bf}
%% The following annotation is customary for chapter which have already been
%% published as a paper.
% \blfootnote{Parts of this chapter have been published in Annalen der Physik \textbf{324}, 289 (1906) \cite{LinWang2011}.}

% %% It is only necessary to list the authors if multiple people contributed
% %% significantly to the chapter.
% %\authors{Albert {\titleshape Einstein}}

% %% The '0pt' option ensures that no extra vertical space follows this epigraph,
% %% since there is another epigraph after it.
% \epigraph[0pt]{
%   A journey of a thousand miles begins with a single step
% }{Laozi}

% \epigraph{
%     Sample quotes
% }{author}

% \begin{abstract}
% Previous researches have shown that different solutions of the optical system can be found using saddle point based method for some simplified cases\cite{PascalTriplet2009}. It is important, however, to study whether the saddle point based method still perform well in practical lens design problems. To study this, we chose to start with a relative simple example.
% \end{abstract}

% %% Start the actual chapter on a new page.
% \newpage
\section{landscape cases}
\section{system specifications}
\item Two systems from the SMS paper, which is in the paper already
\item WAL system from Irina, give the original system is enough.
\item UV microscope system, give the original 
\DUV objective, give the original


\section{math equations}





\references{dissertation}


\documentclass{dissertation}

%% Turn off page numbering for the propositions and make the margins on both
%% sides equal and symmetrical.
\geometry{twoside=false}
\pagestyle{empty}

\begin{document}

%% Specify the title and author of the thesis. This information will be used on
%% both the English and Dutch side and in the metadata of the final PDF.
\title{Systematic Search for New Solutions in Lens Design}
\author{Zhe}{Hou}{哲}{侯}

\begin{center}

{\Large\titlefont\bfseries Propositions}

\bigskip

accompanying the dissertation

\bigskip

%% Print the title.
{\makeatletter
\titlestyle\bfseries\large\@title
\makeatother}


%% Print the optional subtitle.
{\makeatletter
\ifx\@subtitle\undefined\else
    \titlefont\titleshape\@subtitle
\fi
\makeatother}


\bigskip

by

\bigskip

%% Print the full name of the author.
\makeatletter
{\large\titlefont\bfseries\@firstname\ {\titleshape\@lastname}}
\makeatother

\end{center}

\bigskip
\bigskip

\begin{enumerate}

\item Proposition 1.
"Obviously, the quality of the answer is dependent upon the skill of the designer in stating the requirements of the design in an appropriate merit function, as well as selecting a reasonable starting point."\cite{book:Shannon1997}  % I maybe do not need to put a reference
\item Proposition 2.
For the same amount of information, Dutch occupied more pages than English -- 13/13 true cases after comparing the Summary and Samenvatting of those dissertations.  
\item Proposition 3.
\item Proposition 4.
\item Proposition 5.
\item Proposition 6.
\item Proposition 7.
\item Proposition 8.
\item Proposition 9.
\item Proposition 10.

\end{enumerate}

\bigskip
\bigskip

%% Apart from the name and title of the supervisor, the following text is
%% dictated by the promotieregelement.
\begin{center}
These propositions are regarded as opposable and defendable, and have been approved as such by the supervisor prof.\ dr.\ ???.\ ???
\end{center}

\clearpage
{\selectlanguage{dutch}

\begin{center}

{\Large\titlefont\bfseries Stellingen}

\bigskip

behorende bij het proefschrift

\bigskip

%% Print the title.
{\makeatletter
\titlestyle\bfseries\large\@title
\makeatother}

%% Print the optional subtitle.
{\makeatletter
\ifx\@subtitle\undefined\else
    \titlefont\titleshape\@subtitle
\fi
\makeatother}

\bigskip

door

\bigskip

%% Print the full name of the author.
\makeatletter
{\large\titlefont\bfseries\@firstname\ {\titleshape\@lastname}}
\makeatother

\end{center}

\bigskip
\bigskip

\begin{enumerate}

\item Stelling 1.
\item Stelling 2.
\item Stelling 3.
\item Stelling 4.
\item Stelling 5.
\item Stelling 6.
\item Stelling 7.
\item Stelling 8.
\item Stelling 9.
\item Stelling 10.

\end{enumerate}

\bigskip
\bigskip

%% Apart from the name and title of the supervisor, the following text is
%% dictated by the promotieregelement.
\begin{center}
Deze stellingen worden opponeerbaar en verdedigbaar geacht en zijn als zodanig goedgekeurd door de promotor prof.\ dr.\ ???.\ ???.
\end{center}

}

%\bibliography{dissertation}{}
\references{dissertation}


\end{document}


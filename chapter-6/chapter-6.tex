\chapter{Conclusion} %% change
\label{chapter_6} %% change
\graphicspath{ {./chapter-6/figures/} }  %% change
\captionsetup[figure]{labelfont=bf}
\captionsetup{margin=1.5em}
\captionsetup[table]{labelfont=bf}
%% The following annotation is customary for chapter which have already been
%% published as a paper.
\blfootnote{Parts of this chapter have been published in Annalen der Physik \textbf{324}, 289 (1906) \cite{LinWang2011}.}

%% It is only necessary to list the authors if multiple people contributed
%% significantly to the chapter.
%\authors{Albert {\titleshape Einstein}}

%% The '0pt' option ensures that no extra vertical space follows this epigraph,
%% since there is another epigraph after it.
\epigraph[0pt]{
   A journey of a thousand miles begins with a single step
}{Laozi}

\epigraph{
    Sample quotes
}{author}

\begin{abstract}
Previous researches have shown that different solutions of the optical system can be found using saddle point based method for some simplified cases\cite{PascalTriplet2009}. It is important, however, to study whether the saddle point based method still perform well in practical lens design problems. To study this, we chose to start with a relative simple example.
\end{abstract}

%% Start the actual chapter on a new page.
\newpage

Questions:
How many saddle points will result from a scan?
Is the saddle point in the landscape all constructable? With the current technique we do not know how to do that. It is not possible.
Are the minima all connected? via saddle points? - no proof

There are so many things cannot be proven in this context of the research. What we focus is on how to apply the method for solving the lens design problem. 

Pros:
Reduces the amount of search; smooth tradition avoiding ray failures

Cons:
So far only effective in lens design, in the context of adding a pair of curvatures as new variables. With aspheres, it works, but not practical. 
One variable, and with another variable that perfectly compensate the change introduced by the first variable. -->if it is the case, it is a pair of redundant variable, right? 


Dynamics of the landscape
Appearing and disappearing of the saddles and minima. 

There are other complexity in the problem of optimization (slow convergence) has to be considered when applying SPC.

For me this whole picture is a dynamic moving landscape that will change in every unpredictable way. It is utterly difficult to grasp any macro scale structure behind it. 

\references{dissertation}

